\chapter{Conclusion}\label{sec:8}

The present chapter summarises the results of the analysis presented so far and attempts to present a coherent, comprehensive picture of the initial acquisition of a morphosyntactic contrast.

The methodological heart of both the repetition and the comprehension test is the intuition that the processing of inflectional morphology can be studied by manipulating word order: while SO targets can be processed by linking case endings to the corresponding functions or by relying on unmarked word order, only the former strategy will work with OS targets. 

Above-chance accuracy scores in the processing of OS targets, therefore, should represent evidence that the learner has established a solid form–function association between case endings and the corresponding syntactic function. Even though the present study only considered a minimal sub-system of Polish grammar, comprising only two forms and two functions, the VILLA project is not a laboratory experiment exclusively targeting this point, and Polish is not an artificial language: the learner’s task, while apparently easy, had to be performed while breaking into a completely new language, characterised by exotic phonology and vocabulary and dozens of forms and functions to match to each other. That learners should be able to do that was not obvious, and indeed not everyone succeeded in the task. 

The analysis of comprehension errors points to a very clear and not unexpected effect of word order, whereby OS targets are characterised by a much higher error rate, whereas SO targets exhibit a ceiling effect for virtually all learners. These findings are in line the predictions and observations of numerous theoretical frameworks, including Processability Theory (\citealt{Pienemann1998, Baten2013, ArtoniMagnani2015}), Processing Instruction (\citealt{VanPatten1984, VanPatten1996, VanPattenEtAl2013}) and many others (e.g. \citealt{KempeMacWhinney1998, Jackson2007, HenryEtAl2009, Rankin2014}). There is clear processing advantage for SO word order, which is in accordance with its greater diffusion in the languages of the world and indeed in the VILLA participants’ L1s.

The EI task and the semi-spontaneous production task also involve a production component, which makes it possible to observe the principles through which learners attempt to express syntactic relations. While the nominative case in -/a/ is hardly ever repeated incorrectly, the accuracy rate for the accusative case varies greatly. When the latter is not repeated in a target-like manner, it is typically substituted by the -/a/. This appears to be the single, invariable “basic word form” \citep{Perdue1993} for those learners who do not inflect nouns as required by the target sentence. The choice of the ending -a suggests that it is based on NOM case, presumably because of the more favourable distribution of the latter in the input. Indeed, cases in which the invariable word form is based on other inflected forms have been reported in the literature (e.g. \citealt{BroederEtAl1993, GarðarsdóttirÞorvaldsdóttirInPrep}), but this question is beyond the scope of the present work.  

Since the EI test did not include a comprehension or translation component, learner production left some questions unanswered. In order to achieve a clearer picture, the results of the EI task were confronted with those of the comprehension task, which makes it possible to identify a hierarchy of target structure difficulty:

OS repetition ${\supset}$ OS comprehension ${\supset}$ SO repetition ${\supset}$ SO comprehension 

While the two extreme points are quite incontrovertible, the hierarchy seems a little uncertain in its medial compartment, as OS comprehension and SO repetition are correctly processed by a roughly equal number of learners. In any case, the generally tendency shows a facilitatory effect for SO word order and comprehension as opposed to OS and repetition. The reasons behind the learners’ preference for SO have been discussed above; regarding the EI task, it can be argued that it is more complex than the comprehension test in that it encompasses it, while at the same time exerting other demands on the learner. According to most of the literature available, the EI task requires learners not just to repeat a string of sounds, but to decode it (just like in the comprehension test) and then re-produce it, both operations being performed on the basis of the present stage of interlanguage development. 

If one further considers the results of the semi-spontaneous production task, the target-like use of inflectional morphology is even less frequent than in OS repetition, which places this task at the left end of the hierarchy presented above. However, in the production task all transitive utterances have an SO structure. While nothing can thus be said as to the learners’ potential ability to use OS structures in spontaneous production, it is also the case that some learners proved able to accurately use inflectional morphology in OS repetition, but failed to the same in SO structures in the production task. In this respect, the production task is certainly more complex than the EI task because it requires participants to concentrate not only on the form of the message, but also on its content and integration in the interaction. It is not surprising that inflectional morphology should not be given priority \citep{Klein2002}. Despite the absence of inflectional morphology in the speech of most participants, the meaning of utterances is usually retrievable through alternative means, such as semantic and positional principles. The former rely on the fact that the nouns involved in transitive sentences usually differ in their animacy: animate nouns, specifically, have greater probabilities of performing the subject function, while the object function is more likely for inanimate nouns. In the rare cases in which both nouns are animate or inanimate, meaning is retrievable through the default SO word order. This observation too is fully compatible with vast evidence on utterance structure in the early stages of acquisition.

The detailed analysis of the input reveals that OS targets only represent a small subset of all transitive structures, even though the input had been specifically manipulated in order to provide learners with sufficient evidence as to this target structure. Moreover, strong tendencies were found regarding the associations between syntactic functions (hence case endings) with animacy, the vast majority of subjects being instantiated by personal pronouns or person names and most objects being represented by inanimate nouns. Structures with referents not differing in animacy are rare or absent altogether. The targets of the structured tests thus required a certain degree of generalisation involving new lexemes and semantic classes. Due to their object-like semantics, for instance, lexical items like \textit{matematyka}, ‘maths’ only occurred in the input in their accusative form, yet, learners often produced them in an invariable word-form modelled on the nominative case, just like all other lexical items belonging to the same inflectional class. This result adds a precious piece of information to the debate on the factors affecting the choice of the basic word-form of a lexical item, and indeed on the development and complexification of learner varieties on the basis of the input (\citealt{Hulstijn2015}). In the case of the VILLA input, the predominance of the ending [a] in learner output may be explained with reference to its strongest association to the meaning NOM compared to [e] ACC. In terms of construction learning, it thus seems that higher-level, more abstract constructions like the association between case endings and syntactic functions overcome more specific constructions, like that between a given referent and a specific, inflected word form. Alternatively, it cannot be excluded that the preference for [a] may be due to factors somewhat independent of the input, such as the fact that lexical items were always introduced using their citation form (the NOM in [a]).

Turning to the role of the L1, it was predicted that speakers of a morphologically complex language would be facilitated in the processing of a complex target morphological system. This turned out to be the case, as the German learners exhibited overall higher scores in both tests. Cross-linguistic influence turned out to be more complex than hypothesised, though, as the Italian speakers surprisingly performed almost just as well, despite the fact that their language does not express case on full nouns. The key seems to be in the fact that they showed exceptionally good repetition skills, sometimes unrelatedly to the corresponding processing abilities. It could be hypothesised that the Italian lexical stress might play an important role in this respect: while often found on the penultimate syllable, it is in principle free, which in turn could clear the learners from L1-induced bias in segmenting speech. These findings lead to two interesting observations. Firstly, it highlights the importance of perception and perceptual prominence in the (perhaps apparent) processing of morphology (\citealt{GallimoreTharp1981, Peters1985}). Secondly, it raises stimulating doubts as to the nature as well as the validity of the EI task for our research purposes (\citealt{Vinther2002, Erlam2006, VanMoere2012}).

It is now possible to finally outline a fully comprehensive picture of learner processing skills in the earliest hours of SLA. First, a few learners proved able to process inflectional morphology in a structured test after only a few hours of exposure to the input, probably facilitated in this by their L1 (or possibly by other additional languages, e.g. Latin). The amount of input required to reach such results in the structured test is variable, but a small group of participants was able to achieve target-like results by the first test time (9 hours). In contrast, the majority of participants consistently applied a positional principle throughout the experiment in both comprehension and production. All L1 English learners fall within this group, which suggests a clear L1 effect. In between these extreme scenarios, a variety of evolutional patterns may be observed. Results tend to become more target-like over time, which witness to the beneficial effect of further input, although no clear pattern could be identified.

The use of inflectional morphology is rarest in the production task, in which even learners who proved able to successfully process OS targets in comprehension and repetition switch back to a positional mode, in which nouns only exhibit an invariable ending. The lack of functional case marking had no effect on the efficacy of communication, though, as meaning was effectively transmitted through semantic and syntactic means, like animacy contrasts and default word order. It thus appears that while the structured tasks elicited the very best performance which the learner was capable of under laboratory conditions, in which the productive use of the target structure may well emerge. The "actual” competence, i.e. what the learner can do under pressure in a real communicative situation, or otherwise what the speaker \textit{needs} to master in order to be, if not correct, at least effective. Actual production is very likely to have a very different structure from laboratory production, in many respects reflecting that of spontaneous learner varieties.

The main results obtained reported in this book are not revolutionary or surprising \textit{per} \textit{se}. The greater difficulty of production compared to comprehension, differences in grammatical accuracy depending on the task and the existence of marked forms are all facts long acknowledged or at least suspected by both linguistics and language teachers. In this respect, the present work confirms existent observations and brings additional details or domains of application: to name but a few, the acquisition of Slavic languages as L2s have been poorly researched so far, and first exposure studies are only limited to a very short time-span under strictly laboratory conditions. The VILLA project attempted to apply the same rigorous rationale to a communicative situation to a certain extent comparable to existing language teaching practices.

What gives new value to the results presented in this book is precisely the thorough methodology through which they were collected. While the tendencies which emerged from the analysis were mostly known to SLA and language teaching research, the doubt remained that what appeared to be a property of the target structure or a shared acquisitional fact would in fact be due to factors beyond experimental control, among which chiefly the learner’s previous exposure to the target language (as well as to other languages) and input varying in amount and quality. These factors of variability were either eliminated or experimentally controlled in the VILLA project, which makes it possible to focus on the actual acquisitional facts thanks to the reduced disturbance from extra-linguistic factors.

Input control is particularly essential in the debate between nativism and generativism as to its role in the shaping the interlanguage: with respect to the former, the results show that learners do not always conform to the patterns found in the input, but on the contrary are able to generalise them in an innovative way in order to create new structures, perhaps partly reproducing structures belonging to the L1. This is particularly evident in semi-spontaneous production, in which structures occur which are ungrammatical in Polish and as such never occurred in the input. This observation is indeed consistent with the learner variety approach, which shows that learners manipulate the building blocks of input (words and constructions) in a manner that is not always in line with the target language, but that is largely shared cross-linguistically. Again, however, no input control was attempted in these studies, so that this crucial variable inevitably remained a possible source of explanation.

Rather than answer new questions, the present study made it possible to answer existing questions in a more rigorous and comprehensive manner. The wealth of data collected for each learner within the VILLA project describes a rich picture comprising a variety of factors which are not usually found together in a single experiment. Although the present work only used a subset of the data, it conclusions can be further refined or expanded in light of other thoroughly controlled variables. It is hoped that the present analysis made a useful contribution towards the identification of what really matters in SLA, by controlling some of the many variables impacting on each individual learning experience.

\chapter{Pronunciation guide}\label{sec:9}
Below we provide a quick pronunciation guide to Polish standard orthography, useful for reading the examples produced by the native speaker. This section is only intended as a reading aid: for a detailed description of Polish phonology, see \citet{Gussman2007}.

{\centering
\footnotesize
    \begin{tabular}{lllll}
        \lsptoprule
        Orth. & IPA & notes & example & IPA\\
        \midrule
        a & a &  & \textit{ale} "but" & /ale/\\
        ą & ɔN\footnote{The nasal archiphoneme /N/ indicates that the nasal consonant is homorganic with the following segment, and may be realised by its alveolar, bilabial or velar allophones.} & before stops  and affricates & \textit{początek} & /poʧɔntek/\\
        ą & ɔ\~{w} & before fricative and word-final & \textit{mamą} "mum\textsc{[ins]}" & /mamɔ\~{w}/\\
        b & b &  & \textit{niebieski} "blue" & /nʲebo/\\
        c & ts &  & \textit{co} "what" & /tso/\\
        ć & tɕ &  & \textit{pić} "to drink" & /piʨ/\\
        ch & x &  & \textit{dach} "roof" & /dax/\\
        ci & tɕ &  & \textit{ciągnie} "pulls" & /ʨɔŋgnʲe/\\
        cz & tʂ &  & \textit{czarny} "black" & /ʧarnɨ/\\
        d & d &  & \textit{dom} "home" & /dom/\\
        dz & dz &  & \textit{bardzo} "very" & /barʣo/\\
        dź & dʑ &  & \textit{dźwięk} "sound" & /ʥvʲenk/\\
        dż & dʒ &  & \textit{drożdże} "yeast" & /droʒʤe/\\
        dzi & dʑ &  & \textit{gdzie} "where" & /gʥe/\\
        e & ɛ &  & \textit{we} "in" & /ve/\\
        ę & ɛN & before stops and affricates & \textit{między} "between" & /mʲenʣɨ/\\
        ę & ɛ\~{w} & before fricative & \textit{męża} "husband" & /mɛ\~{w}ʒa/\\
        ę & ɛ & word-final & \textit{kawę} "coffee" & /kave/\\
        f & f &  & \textit{flaga} "flag" & /flaga/\\
        g & g &  & \textit{gra} "game" & /gra/\\
        h & x &  & \textit{hotel} "hotel" & /xotel/\\
        i & i &  & \textit{i} "and" & /i/\\
        j & j &  & \textit{ja} "i" & /ja/\\
        l & l &  & \textit{kolor} "colour" & /kolor/\\
        ł & w &  & \textit{mały} "small" & /mawɨ/\\
        m & m &  & \textit{mama} "mum" & /mama/\\
        n & n &  & \textit{komin} "fireplace" & /komin/\\
        ń & ɲ &  & \textit{dzień} "day" & /ʥeɲ/\\
        ni & ɲ &  & \textit{niemiec} "german" & /ɲemʲeʦ/\\
        o & ɔ &  & \textit{to} \textit{"}this" & /to/\\
        ó & u &  & \textit{mówić} "to say" & /muviʨ/\\
        p & p &  & \textit{pokój} "room" & /pokuj/\\
        r & r &  & \textit{rower} "bike" & /rover/\\
        rz & ʒ &  & \textit{dobrze} "good" & /dobʒe/\\
        s & s &  & \textit{jest} "is" & /jest/\\
        ś & ɕ &  & \textit{śpi} "sleeps" & /ɕpi/\\
        si & ɕ &  & \textit{siostra} "sister" & /ɕostra/\\
        sz & ʃ &  & \textit{proszę} "please" & /proʃe/\\
        t & t &  & \textit{tam} "there" & /tam/\\
        u & u &  & \textit{tu} "here" & /tu/\\
        w & v &  & \textit{w} "in" & /v/\\
        y & ɨ &  & \textit{ty} "you" & /tɨ/\\
        ź & ʑ &  & \textit{jeździ} "goes" & /jeʑʥi/\\
        ż & ʒ &  & \textit{mąż} "husband" & /mɔnʃ/\\
        zi & ʑ &  & \textit{zielony} "green" & /ʑelonɨ/\\
        \lspbottomrule
    \end{tabular}
}
\section*{Notes}

\begin{itemize}
    \item Lexical stress always falls on the penultimate syllable, except in learned loanwords from Greek or Latin and when clitics are attached, e.g. \textit{matematyka} "mathematics" /mate'matɨka/; \textit{chodziliśmy} "we went" /xo'ʥiliɕmɨ/; 
    \item Nasals and stops followed by pre-vocalic /i/ are palatalised to various degrees, e.g. \textit{niebieski} "blue" /nʲebʲeski/. The letter <i> in this case effectively functions as a diacritic.
\end{itemize}

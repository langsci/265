\appendix
\addchap{Pronunciation guide}\label{sec:9}
Below we provide a quick pronunciation guide to Polish standard orthography, useful for reading the examples produced by the native speaker. This section is only intended as a reading aid: for a detailed description of Polish phonology, see \citet{Gussman2007}.\bigskip

{\centering
\small
    \begin{tabularx}{\textwidth}{llXll}
        \lsptoprule
        Orth. & IPA & notes & example & IPA\\
        \midrule
        a & a &  & \textit{ale} `but' & [ˈale]\\
        ą & ɔN\footnote{The nasal archiphoneme [N] indicates that the nasal consonant is homorganic with the following segment, and may be realised by its alveolar, bilabial or velar allophones.} & before stops  and affricates & \textit{początek} & [poˈʧɔntek]\\
        ą & ɔ\~{w} & before fricative and word-final & \textit{mamą} `mum\textsc{[ins]}' & [ˈmamɔ\~{w}]\\
        b & b &  & \textit{niebieski} `blue' & [ˈnʲebo]\\
        c & ts &  & \textit{co} `what' & [tso]\\
        ć & tɕ &  & \textit{pić} `to drink' & [piʨ]\\
        ch & x &  & \textit{dach} `roof' & [dax]\\
        ci & tɕ &  & \textit{ciągnie} `pulls' & [ˈʨɔŋgnʲe]\\
        cz & tʂ &  & \textit{czarny} `black' & [ˈʧarnɨ]\\
        d & d &  & \textit{dom} `home' & [dom]\\
        dz & dz &  & \textit{bardzo} `very' & [ˈbarʣo]\\
        dź & dʑ &  & \textit{dźwięk} `sound' & [ʥvʲenk]\\
        dż & dʒ &  & \textit{drożdże} `yeast' & [ˈdroʒʤe]\\
        dzi & dʑ &  & \textit{gdzie} `where' & [gʥe]\\
        e & ɛ &  & \textit{we} `in' & [ve]\\
        ę & ɛN & before stops and affricates & \textit{między} `between' & [ˈmʲenʣɨ]\\
        ę & ɛ\~{w} & before fricative & \textit{męża} `husband' & [ˈmɛ\~{w}ʒa]\\
        ę & ɛ & word-final & \textit{kawę} `coffee' & [ˈkave]\\
        f & f &  & \textit{flaga} `flag' & [ˈflaga]\\
        g & g &  & \textit{gra} `game' & [gra]\\
        h & x &  & \textit{hotel} `hotel' & [ˈxotel]\\
        i & i &  & \textit{i} `and' & [i]\\
        j & j &  & \textit{ja} `i' & [ja]\\
        l & l &  & \textit{kolor} `colour' & [ˈkolor]\\
        ł & w &  & \textit{mały} `small' & [ˈmawɨ]\\
        m & m &  & \textit{mama} `mum' & [ˈmama]\\
        \midrule
    \end{tabularx}
    \newpage
    \begin{tabularx}{\textwidth}{llXll}
        \lsptoprule
        Orth. & IPA & notes & example & IPA\\
        \midrule
        n & n &  & \textit{komin} `fireplace' & [ˈkomin]\\
        ń & ɲ &  & \textit{dzień} `day' & [ʥeɲ]\\
        ni & ɲ &  & \textit{niemiec} `German' & [ˈɲemʲeʦ]\\
        o & ɔ &   & \textit{to} \textit{"}this' & [to]\\
        ó & u &  & \textit{mówić} `to say' & [ˈmuviʨ]\\
        p & p &  & \textit{pokój} `room' & [ˈpokuj]\\
        r & r &  & \textit{rower} `bike' & [ˈrover]\\
        rz & ʒ &  & \textit{dobrze} `good' & [ˈdobʒe]\\
        s & s &  & \textit{jest} `is' & [jest]\\
        ś & ɕ &  & \textit{śpi} `sleeps' & [ɕpi]\\
        si & ɕ &  & \textit{siostra} `sister' & [ˈɕostra]\\
        sz & ʃ &  & \textit{proszę} `please' & [ˈproʃe]\\
        t & t &  & \textit{tam} `there' & [tam]\\
        u & u &  & \textit{tu} `here' & [tu]\\
        w & v &  & \textit{w} `in' & [v]\\
        y & ɨ &  & \textit{ty} `you' & [tɨ]\\
        ź & ʑ &  & \textit{jeździ} `goes' & [ˈjeʑʥi]\\
        ż & ʒ &  & \textit{mąż} `husband' & [mɔnʃ]\\
        zi & ʑ &  & \textit{zielony} `green' & [ˈʑelonɨ]\\
        \lspbottomrule
    \end{tabularx}
}
\section*{Notes}

\begin{itemize}
    \item Lexical stress always falls on the penultimate syllable, except in learned loanwords from Greek or Latin and when clitics are attached, e.g. \textit{matematyka} `mathematics' [mateˈmatɨka]; \textit{chodziliśmy} `we went' [xoˈʥiliɕmɨ];
    \item Nasals and stops followed by pre-vocalic [i] are palatalised to various degrees, e.g. \textit{niebieski} `blue' [nʲˈebʲeski]. The letter <i> in this case effectively functions as a diacritic.
\end{itemize}
